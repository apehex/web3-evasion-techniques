\section{Dynamic Analysis} \label{sec:data-dynamic}

When a transaction is committed to the blockchain, the targeted smart contract is executed.
The actual behavior of the contract can be witnessed first hand in this "dynamic" analysis, rather than infered.

\subsection{Execution Metadata}

First, the execution can be monitored on the blockchain nodes, with the actual live data.

\subsubsection{Transaction Origin}

The records on the blockchain show every address that interacted with a given contract.
Just like the contract's creator, these addresses can be saved and used to correlate different events on the blockchain.

Again, the attackers can answer with lateral movement: new EOAs, new contract instances.

\subsubsection{Transaction Recipient}

Here the \lstinline[language=Solidity]{to} field can only be the contract under inspection.
However it can call other addresses as part of its processing, as seen below.

\subsubsection{Transaction Gas}

As mentioned earlier, gas is directly linked to the intensity of the operations in the transaction.

Like CPU and RAM overloading, intensive computation can be the sign of unwanted activity.
Or it can be exploited for its own value: similarly to CPU / GPU mining, gas can sometimes be redeemed by attackers.

Still, the blockchain always has its "task manager" open, so it is hard to fly these tricks under the radar.

\subsubsection{Transaction Value}

High value transactions are not necessarily bad, but they are bound to attract attention.

Bad actors will lower the noise levels by mixing / scattering the cash flow for example.

\subsection{Event Logs (Topics)}

The events triggered by a given transaction are encoded in the logs, more specifically in their topics and data fields.
The type and arguments of the events hold a lot of information by themselves. Also the emitting address tells what external contracts were called if any.

Sometimes the presence of events is suspicious: in case of a high number of transfers for example.

Other times their absence has implications: upgrading the implementation of a proxy without triggering an \lstinline{Upgraded} event is at least weird.

\subsection{Execution Traces}

Execution traces can be obtained either by replaying locally a transaction or by querying a RPC node with tracing enabled.

\subsubsection{Internal Function Calls}

The flow of internal calls can be debugged locally, which may be the most insightful analysis tool.

Just like traditional malware, smart contracts have means to evade debugging: tests can be detected, the logic of the contract can be cluttered...

\subsubsection{External Function Calls}

A given smart contract can redirect the execution flow to external addresses.
\lstinline[language=Solidity]{address.call} will segregate the contexts of the contracts, while \lstinline[language=Solidity]{address.delegatecall} allows the target contract to modify the state of the origin address.

These external calls may be aimed at:

\begin{itemize}
\item{EOAs, for example to bait them into performing unsafe actions}
\item{legitimate contracts, to loan, launder, exploit, etc}
\item{malicious contracts, to split and layer the suspicious activity}
\end{itemize}

Splitting the logic over several contracts is a way to make local debugging harder too.

\subsection{State Changes}

State changes cover:

\begin{itemize}
\item{modification of the data in the storage slots}
\item{changes to the balance of the address}
\end{itemize}

In particular, the storage of ERC contracts hold a lot of financial information, which is valuable in itself:
token holders, exchange rates, administrative privileges, etc.

Because of the way data is \href{\urldocsstoragelayout}{encoded and positioned in the storage slots}, there is no way to tell which slots are used without context.
This context can come from the transaction history or local debugging.

In any case, the design of the storage makes it stealthy.
