\section{Logic Bomb} \label{sec:logic-bomb}

\subsection{Overview}

As \href{\urlarticlelogicbomb}{Wikipedia states it}: a logic bomb is a piece of code intentionally inserted into a software system that will set off a malicious function when specified conditions are met.
These conditions are usually related to:

\begin{itemize}
\item{the execution time: it can check the \lstinline[language=Solidity]{block.timestamp} or \lstinline[language=Solidity]{block.number} for example}
\item{the execution environment: actually, the technique from section \ref{sec:red-pill} is a subclass of the logic bomb}
\item{patterns in the input data: typically, the execution can depend on the address of the sender}
\end{itemize}

Some logic bombs are meant to counter symbolic testing.
These bombs nest conditional statements without actually caring about the tests themselves.
The simple chaining of conditions has the effect of exponantially increasing the number of execution paths.
In the end, it may overload the testing process.

\subsection{Evasion Targets}

\subsubsection{User Tools}

Just as the red-pill bypassed wallets \ref{sec:red-pill}, logic-bombs may fool other tools.

For example, the past transactions listed in a block explorer may give a false sense of security.
There is no guarantee that similar calls will result in the same results in a different context (different sender, later time, etc).

Honeypots tend to fail once there is enough transaction records to show that the vulnerability is not exploitable.
However, a malicious smart contract may only need to perform it's evil actions in a fraction of the transactions it processes.
These failed attempts could be flooded in attractive promises of gain as shown by other past transactions.

\subsubsection{Security Tools}

Most likely the fuzzing of security tools will remain in the space where the malicious functionalities are disabled.
\href{\urlarticlepathexplosion}{Path explosion} is also designed specifically to break the symbolic analysis of code in general.

\subsection{POC}

\subsection{Detection \& Countermeasures}

\subsubsection{Fuzzing}

Here, the probability of detecting such tricks depends of the extent of the input space covered by the tests.
Security tools should fuzz the metadata of the transactions too.

\subsubsection{Opcodes}

Scanning the bytecode for unusual opcodes may be enough to uncover logic-bombs.
