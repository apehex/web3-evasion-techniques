\section{Logic Bomb} \label{sec:time-bomb}

\subsection{Overview}

As \href{\urlarticlelogicbomb}{Wikipedia states it}: a logic bomb is a piece of code intentionally inserted into a software system that will set off a malicious function when specified conditions are met.
These conditions are usually related to:

\begin{itemize}
\item{the execution time: it can check the \lstinline[language=Solidity]{block.timestamp} or \lstinline[language=Solidity]{block.number} for example}
\item{the execution environment: actually, the technique from section \ref{sec:red-pill} is a subclass of the logic bomb}
\end{itemize}

Some logic bombs are meant to counter symbolic testing.
These bombs nest conditional statements without actually caring about the tests themselves.
The simple chaining of conditions has the effect of exponantially increasing the number of execution paths.
In the end, it may overload the testing process.

\subsection{Evasion Targets}

\subsubsection{Tests}

\subsection{Samples}

\subsubsection{Executing Raw Bytecode}

\subsection{Detection \& Countermeasures}
