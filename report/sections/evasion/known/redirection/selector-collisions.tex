\section{Selector Collisions} \label{sec:selector-collisions}

\subsection{Overview}

Because the function selectors are only 4 bytes long, it is easy to find collisions.

When a selector in the proxy contract collides with another on the implementation side, the proxy takes precedence.

This can be used to override key elements of the implementation.

\subsection{Evasion Targets}

\begin{description}
\item[tools]{this subtle exploit evades most static analysis}
\item[reviewers]{the sources don't show the flow from legitimate function to its malicious collision}
\end{description}

\subsection{Samples}

As \href{https://www.youtube.com/watch?v=l1wjRy2BYPg}{Yoav Weiss showed at DSS 2023} \cite{video-masquerading-code}, this harmless function:

\begin{lstlisting}[language=Solidity]
function IMGURL() public pure returns (bool) {
    return true;
}
\end{lstlisting}

Collides with another function:

\begin{lstlisting}[language=Python]
Web3.keccak(text='IMGURL()').hex().lower()[:10]
# '0xbab82c22'
Web3.keccak(text='vaultManagers(address)').hex().lower()[:10]
# '0xbab82c22'
\end{lstlisting}

And this view is used to determine which address is a manager, e.g. it is critical:

\begin{lstlisting}[language=Solidity]
mapping (address=>bool) public vaultManagers;
\end{lstlisting}

\subsection{Detection \& Countermeasures}

The collisions can be identified by comparing the bytecodes of proxy and implementation:

\begin{description}
\item[selectors]{the hub section of the bytecode has the list of selectors}
\item[debugging]{dynamic analysis will trigger the collision; still it may not have an obviously suspicious behavior}
\end{description}

The article \href{article-deconstructing-contract}{deconstructing a Solidity contract} \cite{article-deconstructing-contract} has a \href{https://gists.rawgit.com/ajsantander/23c032ec7a722890feed94d93dff574a/raw/a453b28077e9669d5b51f2dc6d93b539a76834b8/BasicToken.svg}{very helpful diagram} \cite{image-deconstruction-diagram}.
