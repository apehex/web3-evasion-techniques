\section{Hidden Proxy} \label{sec:hidden-proxy}

\subsection{Overview}

Hidden proxies redirect the execution to another contract just like standard proxies, except that they pretend not to.

Apart from this redirection trick, the rest of the contract code can be anything: a token, ERC-1967 proxy, etc.
There are two cases:

\begin{itemize}
\item{the contract inherits from a reference proxy contract: the expected implementation will serve as a bait and another logic contract is used in practice}
\item{otherwise the delegate contract adds hidden functionalities, like a backdoor}
\end{itemize}

Just like proxies, a common way to achieve this is to performs \lstinline{delegateCall} on any unknown selector, via the fallback function.
In its simplest form, the fallback would just use another address for the logic contract.
More sophiscated attackers will chain proxies or combine this trick with other evasion techniques like \ref{sec:overriding-implementation}.

The target address can be hardcoded or passed as an argument, making it stealthier.

\subsection{Evasion Targets}

This technique stacks another layer of evasion on top those mentioned in \ref{sec:hiding-in-plain-sight}.

\subsubsection{Block Explorers}

Block explorers can detect standard proxy patterns and show the corresponding implementation contract.
In this context, the shown implementation is not used and the explorers are actually misinforming their users.

\subsubsection{Security Tools}

The malicious code is not directly accessible and the tools may end up analysing the legitimate implementation instead.

The actual logic address can be obfuscated or even missing from the bytecode.
Transaction tracing is the most reliable inspection tool in this case, and it is not always available.

\subsection{Samples}

A malicious fallback can be inserted into an expensive codebase:

\begin{lstlisting}[language=Solidity]
fallback () external {
	if (msg.sender == owner()) {
		(bool success, bytes memory data) = address(0x25B072502FB398eb4f428D60D01f18e8Ffa01448).delegateCall(
			msg.data
		);
	}
}
\end{lstlisting}

\subsection{Detection \& Countermeasures}

This attack scheme will most likely involve masquerading code: the indicators mentioned in \ref{sec:hiding-in-plain-sight} provide a basis for the detection.

\subsubsection{Tracing}

The address of the hidden implementation can be found in the trace logs.

\subsubsection{Transaction History}

Comparing the results of transactions over time may show significant differences and highlight a change in implementation.
