\section{Hidden Proxy} \label{sec:hidden-proxy}

\subsection{Overview}

Hidden proxies redirect the execution to another contract just like standard proxies, except that they pretend not to.

Apart from this redirection trick, the rest of the contract code can be anything: a token, ERC-1967 proxy, etc.
There are two cases:

\begin{itemize}
\item{the contract inherits from a reference proxy contract: the expected implementation will serve as a bait and another logic contract is used in practice}
\item{otherwise the delegate contract adds hidden functionalities, like a backdoor}
\end{itemize}

Just like proxies, a common way to achieve this is to performs \lstinline{delegateCall} on any unknown selector, via the fallback function.
In its simplest form, the fallback would just use another address for the logic contract.
More sophiscated attackers will chain proxies or combine this trick with other evasion techniques like \ref{sec:overriding-implementation}.

The target address can be hardcoded or passed as an argument, making it stealthier.

\subsection{Evasion Targets}

This technique stacks another layer of evasion on top those mentioned in \ref{sec:hiding-in-plain-sight}.

\subsubsection{Block Explorers}

Block explorers can detect standard proxy patterns and show the corresponding implementation contract.
In this context, the shown implementation is not used and the explorers are actually misinforming their users.

\subsubsection{Security Tools}

The malicious code is not directly accessible and the tools may end up analysing the legitimate implementation instead.

The actual logic address can be obfuscated or even missing from the bytecode.
Transaction tracing is the most reliable inspection tool in this case, and it is not always available.

\subsection{Samples}

A malicious fallback can be inserted into an expensive codebase:

\begin{lstlisting}[language=Solidity]
fallback () external {
    if (msg.sender == owner()) {
        (bool success, bytes memory data) = address(0x25B072502FB398eb4f428D60D01f18e8Ffa01448).delegateCall(
            msg.data
        );
    }
}
\end{lstlisting}

\subsection{Detection \& Countermeasures}

This attack scheme will most likely involve masquerading code: the indicators mentioned in \ref{sec:hiding-in-plain-sight} provide a basis for the detection.

\subsubsection{Overall process}

\hspace*{-1.5cm}
\begin{tikzpicture}

% \node (metadata)    [io] {Transaction metadata};
% \node (preprocess)  [block, below=1cm of metadata] {Preprocess};
\node (abi)         [io] {ABI, opcodes, selector}; % below=1cm of preprocess

\node (delegation) [decision, below=1cm of abi] {Delegation?};
\node (proxy) [decision, below=1cm of delegation] {Std. proxy?};
\node (logic) [block, below=1cm of proxy] {Extract standard addresses};
\node (called) [block, below=1cm of logic] {Extract called address};
\node (match) [decision, below=1cm of called] {Match?};
\node (positive) [io, below=2cm of match] {1};

\node (negative) [io, right=3cm of delegation] {0};

\node (token) [decision, left=2cm of proxy] {Token?};
\node (unknown) [io, left=3cm of token] {0.5};
\node (almost) [io, below=8.2cm of token] {0.9};

\node (c1) [container, fit=(delegation) (match) (token), inner xsep=22mm, inner ysep=8mm, xshift=2mm, yshift=-4mm] {};
\node (t1) [label, above=-1.5cm of c1, xshift=-4cm] {Hidden proxy?};

\node (c2) [container, fit=(unknown)] {};
\node (t2) [label, above=2mm of c2] {Unknown};

\node (c3) [container, fit=(almost) (positive)] {};
\node (t3) [label, above=-1cm of c3] {Positive};

\node (c4) [container, fit=(negative)]{};
\node (t4) [label, above=2mm of c4] {Negative};

% \draw [arrow] (metadata) -- (preprocess);
% \draw [arrow] (preprocess) -- (abi);
\draw [arrow] (abi) -- (delegation);
\draw [arrow] (delegation) -- node[label, near start, right] {yes, 0.5} (proxy);
\draw [arrow] (delegation) -- node[label, near start, above] {no, 0} (negative);
\draw [arrow] (proxy) -- node[label, near start, right] {yes, 0.5} (logic);
\draw [arrow] (proxy) -- node[label, near start, above] {no, 0.5} (token);
\draw [arrow] (logic) -- (called);
\draw [arrow] (called) -- (match);
\draw [arrow] (token) --  node[label, near start, above] {no, 0.5} (unknown);
\draw [arrow] (token) --  node[label, near start, left] {yes, 0.9} (almost);
\draw [arrow] (match) -| node[label, near start, above] {yes, 0} (negative);
\draw [arrow] (match) -- node[label, near start, right] {no, 1} (positive);

\end{tikzpicture}


This scheme is restricted to two subclasses of the hidden-proxies: standard proxies that don't follow the specifications and tokens that act as proxies.
It can be extended and improved upon.

\subsubsection{Delegation}

Delegation can be detected by comparing the selector from the transaction data with contract's interface.

The contract's interface itself can be extracted from the bytecode, in the hub section of the contract.

\subsubsection{Standard Proxy \& Token}

Once the interface is extracted from the bytecode, it can be compared with known interfaces.
In particular tokens and proxies have well-known and constant interfaces.

\subsubsection{Implementation addresses}

The implementation address can be retrieved from the storage of standard addresses.
It is stored at a fixed slot for each standard:

\begin{lstlisting}[language=Python]
LOGIC_SLOTS = {
    # bytes32(uint256(keccak256('eip1967.proxy.implementation')) - 1)
    'erc-1967': '360894a13ba1a3210667c828492db98dca3e2076cc3735a920a3ca505d382bbc',
    # keccak256("org.zeppelinos.proxy.implementation")
    'zeppelinos': '7050c9e0f4ca769c69bd3a8ef740bc37934f8e2c036e5a723fd8ee048ed3f8c3',
    # keccak256("PROXIABLE")
    'erc-1822': 'c5f16f0fcc639fa48a6947836d9850f504798523bf8c9a3a87d5876cf622bcf7',}
\end{lstlisting}

The address to which the transaction call was redirected can be identified in the traces.
Or it can be parsed from the bytecode and / or the transaction data depending on the logic of the contract.
