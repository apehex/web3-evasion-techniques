\section{Event Poisoning} \label{sec:event-poisoning}

\subsection{Overview}

Events have the underlying implication that some change happened and the blockchain state evolved.

It is actually possible to trigger events without their side effects.
For instance, by setting the amount to 0 on the ERC20 \lstinline[language=Solidity]{tranfer} it is possible to trigger \lstinline[language=Solidity]{Transfer} events without moving any token!

Actually, all standards and events could potentially be hijacked.

\subsection{Evasion Targets}

\subsubsection{Users}

Many users don't double check events, especially not when they come from well-known tokens / contracts.

\subsection{Samples}

In \href{https://explorer.phalcon.xyz/tx/polygon/0x8a5f75338bfbf78b0969cdf5bacfe24c65e703ea94b430c470193b3d2a094441?line=1}{this batch transaction}, the scammer pretended to send USDC, DAI and USDT to 12 addresses.
The attacker baited users by coupling two transfers:

\begin{itemize}
\item{a transfer of 0 amount of a popular token, say USDT}
\item{a transfer of a small amount of a fake token, with the same name and symbol}
\end{itemize}

The Forta network \href{https://explorer.forta.network/alerts?limit=20&sort=desc&text=&txHash=0x8a5f75338bfbf78b0969cdf5bacfe24c65e703ea94b430c470193b3d2a094441}{detected the transfer events of null amount}.

\subsection{Detection \& Countermeasures}

\subsubsection{Event Logs}

These scams are easily uncovered by parsing the transaction logs.
They contain the list of events, with arguments that specify what happened.

The idea is to define constraints on the arguments of all the standard events.
For example, valid ERC20 \lstinline{Transfer} events would have a constraint \emph{strictly greater than zero} on the \lstinline{amount} argument.

Then these constraints would be checked on the live transactions.

Ideally, these checks should be performed beforehand, by the contracts behind the event emission.
