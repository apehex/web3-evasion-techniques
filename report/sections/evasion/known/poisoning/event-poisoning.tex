\section{Event Poisoning} \label{sec:event-poisoning}

\subsection{Overview}

Events have the underlying implication that some change happened and the blockchain state evolved accordingly.

It is actually possible to trigger events without their side effects or with mismatching effects.
For instance, by setting the amount to 0 on the ERC20 \lstinline[language=Solidity]{tranfer} it is possible to trigger \lstinline[language=Solidity]{Transfer} events without moving any token!

Actually, all standards and events could potentially be hijacked.

\subsection{Evasion Targets}

\subsubsection{Users}

Many users don't double check events, especially not when they come from well-known tokens / contracts.

\subsection{Samples}

In \href{https://explorer.phalcon.xyz/tx/polygon/0x8a5f75338bfbf78b0969cdf5bacfe24c65e703ea94b430c470193b3d2a094441?line=1}{this batch transaction}, the scammer pretended to send USDC, DAI and USDT to 12 addresses.
The attacker baited users by coupling two transfers:

\begin{itemize}
\item{a transfer of 0 amount of a popular token, say USDT}
\item{a transfer of a small amount of a fake token, with the same name and symbol}
\end{itemize}

The Forta network \href{https://explorer.forta.network/alerts?limit=20&sort=desc&text=&txHash=0x8a5f75338bfbf78b0969cdf5bacfe24c65e703ea94b430c470193b3d2a094441}{detected the transfer events of null amount}.

\pagebreak
\subsection{Detection \& Countermeasures}

\subsubsection{Overall process}

\hspace*{-2cm}
\begin{tikzpicture}

% \node (metadata)    [io] {Transaction metadata};
% \node (preprocess)  [block, below=1cm of metadata] {Preprocess};
\node (abi)         [io] {ABI, opcodes, selector}; % below=1cm of preprocess

\node (delegation) [decision, below=1cm of abi] {Delegation?};
\node (proxy) [decision, below=1cm of delegation] {Std. proxy?};
\node (logic) [block, below=1cm of proxy] {Extract standard addresses};
\node (called) [block, below=1cm of logic] {Extract called address};
\node (match) [decision, below=1cm of called] {Match?};
\node (positive) [io, below=2cm of match] {1};

\node (negative) [io, right=3cm of delegation] {0};

\node (token) [decision, left=2cm of proxy] {Token?};
\node (unknown) [io, left=3cm of token] {0.5};
\node (almost) [io, below=8.2cm of token] {0.9};

\node (c1) [container, fit=(delegation) (match) (token), inner xsep=22mm, inner ysep=8mm, xshift=2mm, yshift=-4mm] {};
\node (t1) [label, above=-1.5cm of c1, xshift=-4cm] {Hidden proxy?};

\node (c2) [container, fit=(unknown)] {};
\node (t2) [label, above=2mm of c2] {Unknown};

\node (c3) [container, fit=(almost) (positive)] {};
\node (t3) [label, above=-1cm of c3] {Positive};

\node (c4) [container, fit=(negative)]{};
\node (t4) [label, above=2mm of c4] {Negative};

% \draw [arrow] (metadata) -- (preprocess);
% \draw [arrow] (preprocess) -- (abi);
\draw [arrow] (abi) -- (delegation);
\draw [arrow] (delegation) -- node[label, near start, right] {yes, 0.5} (proxy);
\draw [arrow] (delegation) -- node[label, near start, above] {no, 0} (negative);
\draw [arrow] (proxy) -- node[label, near start, right] {yes, 0.5} (logic);
\draw [arrow] (proxy) -- node[label, near start, above] {no, 0.5} (token);
\draw [arrow] (logic) -- (called);
\draw [arrow] (called) -- (match);
\draw [arrow] (token) --  node[label, near start, above] {no, 0.5} (unknown);
\draw [arrow] (token) --  node[label, near start, left] {yes, 0.9} (almost);
\draw [arrow] (match) -| node[label, near start, above] {yes, 0} (negative);
\draw [arrow] (match) -- node[label, near start, right] {no, 1} (positive);

\end{tikzpicture}


\subsubsection{Parsing the logs}

First the logs have to be parsed and decoded from the transaction topics and data.
Like functions, events have selectors which can be reversed with rainbow tables.

\subsubsection{Constraints on the events}

The idea is to define constraints on the arguments of all the standard events.
For example, valid ERC20 \lstinline{Transfer} events would have a constraint \emph{strictly greater than zero} on the \lstinline{amount} argument.
A change to the implementation address of a proxy should trigger a \lstinline{Upgraded} event, etc.

Here the decision block has a black \& white output, but in reality it would be a probability.
This probability depends on the event and the constraint that was broken, so it cannot be shown in this diagram as a generic metric for all the cases.

The overall output of the process is the \href{https://www.ams.org/journals/tran/2011-363-06/S0002-9947-2011-05340-7/S0002-9947-2011-05340-7.pdf}{conflation of the probabilities on each event}.

\subsubsection{Building the referential}

The reference is a database indexing the selectors of all the standards and matching them with the event signature.
