\section{Red-Pill} \label{sec:red-pill}

\subsection{Overview}

The red-pill technique detects simulation environment to disable its exploits upon scrutiny. 

The contract detects simulation environments by checking:

\begin{description}
\item[globals]{these variables have special values in test environments:
\begin{itemize}
    \item{\lstinline[language=Solidity]{block.basefee}: \lstinline{0}}
    \item{\lstinline[language=Solidity]{block.coinbase}: \lstinline{0x0000000000000000000000000000000000000000}}
    \item{\lstinline[language=Solidity]{tx.gasprice}: large numbers, higher than \lstinline{0xffffffffffffffff}}
\end{itemize}}
\end{description}

Then it triggers legitimate code in simulation contexts and malicious code on the mainnet.

\subsection{Evasion Targets}

\subsubsection{Wallets}

Wallets often perform a simulation of the transaction before committing.
The whole point of this method is to pass these tests and bait the end-user.

\subsubsection{Security Tools}

Automatic tools will likely not fuzz the coinbase or other global variables.
So the dynamic analysis may follow the "harmless" branch and not inspect the actual behavior of the contract on the mainnet.

On the other hand these unusual checks stand out when reviewing the code.

\subsection{Samples}

The contract \lstinline{FakeWethGiveaway} mentioned in \href{\urlarticleredpill}{the Zengo article} checks the current block miner’s address:

\begin{lstlisting}[language=Solidity]
function checkCoinbase() private view returns (bool result) {
    assembly {
        result := eq(coinbase(), 0x0000000000000000000000000000000000000000)
    }
}
\end{lstlisting}

When null (test env), it actually sends a reward:

\begin{lstlisting}[language=Solidity]
bool shouldDoTransfer = checkCoinbase();
if (shouldDoTransfer) {
    IWETH(weth).transfer(msg.sender, IWETH(weth).balanceOf(address(this)));
}
\end{lstlisting}

Otherwise, on the mainnet, it just accepts transfers without doing anything.

\subsection{Detection \& Countermeasures}

\subsubsection{Bytecode}

The bytecode can be disassembled and scanned for unusual opcodes: typically \lstinline[language=Solidity]{block.coinbase}.

\subsubsection{fuzzing}

The transactions can be tested with blank data and compared with the historic transactions.
