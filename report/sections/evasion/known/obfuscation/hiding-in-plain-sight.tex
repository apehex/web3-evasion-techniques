\section{Hiding In Plain Sight} \label{sec:hiding-in-plain-sight}

\subsection{Overview}

By stacking dependencies, the scammer grows the volume of the source code to thousands of lines.

99\% of the code is classic, legitimate implementation of standards.

And the remaining percent is malicious code: it can be in the child class or hidden inside one of the numerous dependencies.

This technique is the most basic: it is often used in combination with other evasion methods.

\subsection{Evasion Targets}

\subsubsection{Code Reviewers}

A single line can compromise the whole codebase, so the reviewing process is very laborious and slow.
Attackers stuff the code to overwhelm security auditors with the sheer volume of code.

\subsubsection{Security Tools}

Unrelated data also lowers the efficiency of ML algorithms:
adding valid code will increase the chances of the contract to be classified as harmless.

\subsection{Samples}

Hidden among 7k+ lines of code:

\begin{lstlisting}[language=Solidity]
// no authorization modifier `onlyOwner`
function transferOwnership(address newOwner) public virtual {
    if (newOwner == address(0)) {
        revert OwnableInvalidOwner(address(0));
    }
    _transferOwnership(newOwner);
}
\end{lstlisting}

\subsection{Detection \& Countermeasures}

\subsubsection{Bytecode}

The size of the bytecode is a low signal, but:

\begin{itemize}
\item{it is easy to detect, with certainty}
\item{the codebase is always large when this technique is used}
\item{not all contracts are bulky, some can be filtered out}
\end{itemize}

The tricky part is to choose a relevant threshold on the size of the code.

\subsubsection{Execution Traces}

The proportion of the code actually used can be computed by replaying transactions.

It is important to \emph{replay the past transactions} and \textbf{not} perform new tests.
Indeed, testing all the functions would skew the statistics on mainnet usage.
