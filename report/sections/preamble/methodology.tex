\section{Methodology} \label{sec:methodology}

This state-of-the-art is grounded in both past and present research.

A literature review on traditional malware evasion forms the basis for the study's taxonomy and framework.
Studying these historical evasion techniques gives insights into potential trends for the blockchain ecosystem.

In addition to the lessons from the past, the study also incorporates findings from current research in the web3 space.
This research is sourced from academic papers, conferences, tools, and watch groups focused on blockchain security.

The report's practical aspect is backed by an analysis of selected smart contract samples.
These samples were chosen for two reasons: their association with recent hacks and their ability to slip past detection mechanisms, especially those of the \href{https://explorer.forta.network/}{Forta network}.
Contracts from other platforms such as \href{https://www.chainabuse.com/reports}{chainabuse}, \href{https://www.web3rekt.com/}{web3rekt}, and \href{https://www.rekt.news/}{rekt.news} provided the necessary data for this analysis.
