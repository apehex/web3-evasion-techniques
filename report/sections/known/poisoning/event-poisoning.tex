\section{Event Poisoning} \label{sec:event-poisoning}

\subsection{Overview}

By setting the amount to 0, it is possible to trigger \lstinline[language=Solidity]{Transfer} events from any ERC20 contracts.

In particular, scammers  bait users by coupling two transfers:

\begin{itemize}
\item{a transfer of 0 amount of a popular token, say USDT}
\item{a transfer of a small amount of a fake token, with the same name and symbol}
\end{itemize}

\subsection{Evasion Targets}

\begin{description}
\item[users]{many users don't double check events coming from well-known tokens}
\end{description}

\subsection{Samples}

In \href{https://explorer.phalcon.xyz/tx/polygon/0x8a5f75338bfbf78b0969cdf5bacfe24c65e703ea94b430c470193b3d2a094441?line=1}{this batch transaction}, the scammer pretended to send USDC, DAI and USDT to 12 addresses.

The Forta network \href{https://explorer.forta.network/alert/0x51add5ade0777f3fd65efb97ea0055aa6a5329bcfa8266e11c9de28da81896d7}{detected the transfer events of null amount}.

\subsection{Detection \& Countermeasures}

These scams are easily uncovered:

\begin{description}
\item[logs]{the transactions logs contain the lsit of events, whose amounts can be parsed}
\end{description}