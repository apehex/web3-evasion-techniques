\section{Red-Pill} \label{sec:red-pill}

The red-pill technique detects simulation environment to disable its exploits upon scrutiny. 

\subsection{Evades}

Live tests in transaction simulations: often performed by wallets before sending a transaction.

\subsection{How}

The contract detects simulation environments by:

\begin{itemize}
\item{comparing the global variables with settings found in simulated environments:
\begin{itemize}
    \item{\emph{block.basefee} with}
    \item{\emph{block.coinbase} with \emph{0x0000000000000000000000000000000000000000}}
    \item{\emph{tx.gasprice} with}
\end{itemize}}
\end{itemize}

Then it triggers legitimate code in simulation contexts and malicious code on the mainnet.

\subsection{Samples}

The contract [FakeWethGiveaway](red-pill/FakeWethGiveaway.sol) checks the current block miner’s address:

\begin{lstlisting}[language=Solidity]
function checkCoinbase() private view returns (bool result) {
    assembly {
        result := eq(coinbase(), 0x0000000000000000000000000000000000000000)
    }
}
\end{lstlisting}

When null (test env), it actually sends a reward and otherwise it just accepts transfers without doing anything.s

\subsection{Detection \& Countermeasures}

\begin{itemize}
\item{Looking for unusual opcodes: typically `block.coinbase`.}
\item{Replaying transactions and fuzzing the global variables.}
\end{itemize}

\subsection{Resources}

\begin{itemize}
\item{\cite{article-red-pill}}
\end{itemize}
