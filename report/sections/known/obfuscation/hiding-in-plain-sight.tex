\section{Hiding In Plain Sight} \label{sec:hiding-in-plain-sight}

\subsection{Technical Details}

By stacking dependencies, the scammer grows the volume of the source code to thousands of lines.

99\% of the code is classic, legitimate implementation of standards.

And the remaining percent is malicious code: it can be in the child class or hidden inside one of the numerous dependencies.

\subsection{Evasion Targets}

\begin{description}
\item[users]{wallets often perform a simulation of the transaction before committing}
\item[reviewers]{the goal is to overwhelm source code reviewers with the sheer volume of code}
\item[tools]{unrelated data also lowers the efficiency of ML algorithms}
\end{description}

\subsection{Samples}

Hidden among 7k+ lines of code:

\begin{lstlisting}
// no authorization modifier `onlyOwner`
function transferOwnership(address newOwner) public virtual {
    if (newOwner == address(0)) {
        revert OwnableInvalidOwner(address(0));
    }
    _transferOwnership(newOwner);
}
\end{lstlisting}

\subsection{Detection \& Countermeasures}

\begin{description}
\item[bytecode]{the size of the bytecode is a low signal}
\item[tracing]{the proportion of the code actually used can be computed by replaying transactions}
\end{description}
