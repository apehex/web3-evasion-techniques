\section{Overriding Standards Implementation}

\subsection{Technical Details}

Like the previous technique \ref{sec:fake-implementation}, the goal is to have a malicious contract confused with legitimate code.

It is achieved by inheriting from standardized code like \lstinline{Ownable}, \lstinline{Upgradeable}, etc.
Then, the child class overwrites key elements with:

\begin{description}
\item[redefinition]{an existing keyword is defined a second time for the references in the child class only}
\item[polymorphism]{an existing method can be redined with a slightly different signature}
\end{description}

From the perspective of the source code, a single keyword like \lstinline{owner} can refer to different storage slot depending on its context.
It is only in the bytecode that a clear difference is made.

\subsection{Evasion Targets}

This technique is a refinment of the previous one: it will work on more targets.

\begin{description}
\item[Etherscan]{blockchain explorers lack  even more flexibility to detect these exploits}
\item[Users]{the source code is even closer to a legitimate contract}
\item[Reviewers]{the interpretation of the source code is subtle, and reviewing the bytecode is very time consuming}
\end{description}

\subsection{Samples}

\subsubsection{Attribute Overwriting}

In section \emph{3.2.2}, the paper \cite{paper-art-of-the-scam} shows an example of inheritance overriding with \lstinline{KingOfTheHill} :

\begin{lstlisting}[language=Solidity]
contract KingOfTheHill is Ownable {
    address public owner; // different from the owner in Ownable

    function () public payable {
        if(msg.value > jackpot) owner = msg.sender; // local owner
        jackpot += msg.value;
    }
    function takeAll () public onlyOwner { // contract creator
        msg.sender.transfer(this.balance);
        jackpot = 0;
    }
}
\end{lstlisting}

In the modifier on \lstinline{takeAll}, the \lstinline{owner} points to the contract creator.
It is at storage slot 1, while the fallback function overwrites the storage slot 2.

In short, sending funds to this contract will never make you the actual owner.

\subsubsection{Method Overwriting}

\begin{lstlisting}[language=Solidity]

\end{lstlisting}

\newpage
\subsection{Detection \& Countermeasures}

While subtle for the human reader, tools can rather easily detect it in:

\begin{description}
\item[source code]{the sources can be checked for duplicate definitions \& polymorphism}
\item[bytecode]{}
\end{description}

Since the whole point is to advertize for a functionality with the sources, they will be available.
