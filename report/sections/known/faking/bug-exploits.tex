\section{Bug Exploits}

\subsection{Technical Details}

A more vicious way to mask ill-intented code is to exploit bugs and EVM quirks.

By definition, these bugs trigger unwanted / unexpected behaviors.

They can be:

\begin{description}
\item[EVM quirks]{in particular, some operations are implied and not explicitely written}
\item[bugs]{the Solidity language itself has \href{https://github.com/ethereum/solidity/blob/develop/docs/bugs.json}{numerous bugs}, depending on the version used at compilation time \cite{changelog-solidity-bugs}}
\end{description}

They are usually leveraged in honeypots, where the attackers create a contract that looks vulnerable.
But the "vulnerability" doesn't work and people who try to take advantage of it will lose their funds.

\subsection{Evasion Targets}

\begin{description}
\item[tools]{honeypots are meants to trigger alerts in popular tools and mislead their users}
\item[reviewers]{successfully used in honeypots, these tricks can fool security professional}
\end{description}

\subsection{Samples}

\subsubsection{Impossible Conditions}

Attackers can craft a statement that will never be true.

A \href{https://www.youtube.com/watch?v=4bSQWoy5a_k}{minimal example} was given at DEFI summit 2023 by Noah Jelic \cite{video-hacker-traps}:

\begin{lstlisting}[language=Solidity]
function multiplicate() payable external {
    if(msg.value>=this.balance) {
        address(msg.sender).transfer(this.balance+msg.value);
    }
}
\end{lstlisting}

This gives the illusion that anyone may-be able to withdraw the contract's balance.

However, at the moment of the check, \lstinline[language=Solidity]{this.balance} has already been incremented: it can never be lower than \lstinline[language=Solidity]{msg.value}.

In reality, the contract would have exactly the same behavior if the \lstinline{multiplicate} function was empty.

\subsection{Detection \& Countermeasures}

\begin{description}
\item[testing]{symbolic testing \& fuzzing will show the actual behavior; the issue is rather to formulate what is expected for any arbitrary contract}
\item[CVEs]{known vulnerabilities can be identified with pattern matching; in traditional malware detection, \href{https://yara.readthedocs.io/en/stable/writingrules.html}{YARA rules} are written}
\end{description}

There's a tool aimed specifically at detecting honeypots, \href{https://github.com/christoftorres/HoneyBadger}{HoneyBadger}.
