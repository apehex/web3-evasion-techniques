\section{Inheritance Overriding}

\subsection{Evades}

Source code reviews with subtle exploitation of the compilation process.

\subsection{How}

The malicious contract inherits from standard code like \emph{Ownable}, \emph{Upgradeable}, etc.

It overwrites key elements by:

\begin{itemize}
\item{adding a variable definition for an existing keyword}
\item{polymorphism, which allows to have several versions of a function}
\end{itemize}

Then a single keyword can refer to different implementations depending on its context.

The resulting contract doesn't behave like its parent, while looking legitimate.

\subsection{Samples}

\subsubsection{Attribute Overwriting}

\emph{KingOfTheHill} inherits from \emph{Ownable} but the original \emph{owner} cannot be changed:

\begin{lstlisting}[language=Solidity]
contract KingOfTheHill is Ownable {
    address public owner; // different from the owner in Ownable

    function () public payable {
        if(msg.value > jackpot) owner = msg.sender; // local owner
        jackpot += msg.value;
    }
    function takeAll () public onlyOwner { // owner from Ownable = contract creator
        msg.sender.transfer(this.balance);
        jackpot = 0;
    }
}
\end{lstlisting}

In the modifier on \emph{takeAll}, the \emph{owner} points to the contract creator.
It is at storage slot 1, while the fallback function overwrites the storage slot 2.

In short, sending funds to this contract will never make you the actual owner.

\subsubsection{Method Overwriting}

\begin{lstlisting}

\end{lstlisting}

\subsection{Detection \& Countermeasures}

\begin{itemize}
\item{Caveat: these overrides appear in the sources but not in the bytecode.}
\item{The sources can be checked for duplicate definitions / polymorphism.}
\end{itemize}

Since the whole point is to advertize for a functionality with the sources, they will be available.

\subsection{Resources}

\begin{itemize}
\item{\cite{paper-art-of-the-scam}, section \emph{3.2.2}}
\item{\cite{video-masquerading-code}}
\end{itemize}
