\usepackage{tikz} % Required for drawing custom shapes
\usetikzlibrary{arrows, backgrounds, calc, fit, mindmap, positioning, shapes, shapes.geometric, tikzmark}

%----------------------------------------------------------------------------------------
%   GEOMETRY
%----------------------------------------------------------------------------------------

\newcommand{\drawhexagon}[5]{
    % #1 - text
    % #2 - position
    % #3 - size
    % #4 - rotation
    % #5 - options
    \node[rounded corners, inner sep=0, ultra thick, regular polygon, regular polygon sides=6, minimum size=#3, rotate=#4, #5] at (#2) {
        {\sffamily \bfseries #1}
    }
}

\newcommand{\drawtext}[5]{
    % #1 - text
    % #2 - position
    % #3 - height
    % #4 - rotation
    % #5 - options
    \node[left, bg, rounded corners, minimum width=0.5*\paperwidth, minimum height=#3, text width=16cm, execute at begin node=\setlength{\baselineskip}{2em}, rotate=#4, #5] at (#2){
        {\sffamily \bfseries #1}
    }
}

\newcommand\pentagonvertex[5]{
    % #1 - Ox
    % #2 - Oy
    % #3 - R, radius of the outer circle including the vertexes
    % #4 - Theta, the tilt angle
    % #5 - I, index of the vertex
    ({#1 + #3*cos(72*#5 + #4)},%
     {#2 + #3*sin(72*#5 + #4)})%
}

\newcommand{\drawfortalogo}[7]{
    % #1 - Ox
    % #2 - Oy
    % #3 - R, radius of the outer circle including the vertexes
    % #4 - Alpha, the angle between consecutive vertexes on a branch (eg. the angular width)
    % #5 - Beta, the curvatur angle, as a deviation from the direct line between nodes
    % #6 - Theta, the tilt angle
    % #7 - Styling options
    % \draw[dotted]   \pentagonvertex{#1}{#2}{0.4*#3}{-36}{0} --
    %                 \pentagonvertex{#1}{#2}{0.4*#3}{-36}{1} --
    %                 \pentagonvertex{#1}{#2}{0.4*#3}{-36}{2} --
    %                 \pentagonvertex{#1}{#2}{0.4*#3}{-36}{3} --
    %                 \pentagonvertex{#1}{#2}{0.4*#3}{-36}{4} -- cycle;

    % \draw[dotted, draw=blue]    \pentagonvertex{#1}{#2}{#3}{-0.5*#4}{0} --
    %                     \pentagonvertex{#1}{#2}{#3}{-0.5*#4}{1} --
    %                     \pentagonvertex{#1}{#2}{#3}{-0.5*#4}{2} --
    %                     \pentagonvertex{#1}{#2}{#3}{-0.5*#4}{3} --
    %                     \pentagonvertex{#1}{#2}{#3}{-0.5*#4}{4} -- cycle;

    % \draw[dotted, draw=red]     \pentagonvertex{#1}{#2}{#3}{0.5*#4}{0} --
    %                     \pentagonvertex{#1}{#2}{#3}{0.5*#4}{1} --
    %                     \pentagonvertex{#1}{#2}{#3}{0.5*#4}{2} --
    %                     \pentagonvertex{#1}{#2}{#3}{0.5*#4}{3} --
    %                     \pentagonvertex{#1}{#2}{#3}{0.5*#4}{4} -- cycle;

    \foreach \i [evaluate=\i as \j using {int(mod(\i+1,5))},
                evaluate=\i as \Aout using {#6 + 90 + #5 + \i*72},
                evaluate=\i as \Bin using {#6 - 90 - #5 + \i*72},
                evaluate=\i as \Bout using {#6 + 180 + #5 + \i*72},
                evaluate=\i as \Cin using {#6 - #5 + \i*72},
                evaluate=\i as \Cout using {#6 + 72 + #5 + \i*72},
                evaluate=\i as \Din using {#6 + 252 - #5 + \i*72}
               ] in {0,1,2,3,4} {
        \path let \p1 = \pentagonvertex{#1}{#2}{#3}{#6-0.5*#4}{\i},
                  \p2 = \pentagonvertex{#1}{#2}{#3}{#6+0.5*#4}{\i},
                  \p3 = \pentagonvertex{#1}{#2}{0.4*#3}{#6+36}{\i},
                  \p4 = \pentagonvertex{#1}{#2}{#3}{#6-0.5*#4}{\j}
              in coordinate (A) at (\p1)
                 coordinate (B) at (\p2)
                 coordinate (C) at (\p3)
                 coordinate (D) at (\p4);
        
        \draw[#7] (A) to[out=\Aout,in=\Bin] (B) 
                  to[out=\Bout,in=\Cin] (C) 
                  to[out=\Cout,in=\Din] (D);
    }
}

%----------------------------------------------------------------------------------------
%   NODE TYPES
%----------------------------------------------------------------------------------------

\tikzstyle{title} = [
    minimum width=3cm,
    draw=none,
    text=fg,
    font=\bf,
    inner sep=0pt]

\tikzstyle{line} = [
    draw=border,
    -latex',
    very thick]

\tikzstyle{arrow} = [
    ->,
    >=stealth,
    draw=border,
    very thick]

\tikzstyle{label} = [
    text centered,
    text=fg,
    very thick]

\tikzstyle{startstop} = [
    rectangle,
    rounded corners,
    text centered,
    draw=border,
    fill=bg,
    text=fg,
    very thick,
    inner sep=2mm]

\tikzstyle{io} = [
    trapezium,
    trapezium left angle=70,
    trapezium right angle=110,
    text centered,
    draw=border,
    fill=bg,
    text=fg,
    very thick,
    inner sep=2mm]

\tikzstyle{container} = [
    rectangle,
    draw=border,
    dashed,
    very thick,
    text=fg,
    inner sep=4mm]

\tikzstyle{block} = [
    rectangle,
    text centered,
    draw=border,
    fill=bg,
    text=fg,
    very thick,
    inner sep=2mm]

\tikzstyle{decision} = [
    diamond,
    aspect=2,
    text centered,
    draw=border,
    fill=bg,
    text=fg,
    very thick,
    inner sep=2mm]
