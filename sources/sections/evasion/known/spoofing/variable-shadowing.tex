\section{Variable Shadowing} \label{sec:variable-shadowing}

\subsection{Overview}

Like the previous technique \ref{sec:fake-implementation}, the goal is to have a malicious contract confused with legitimate code.

It is achieved by inheriting from standardized code like \lstinline{Ownable}, \lstinline{Upgradeable}, etc.
Then, the child class overwrites key elements with:

\begin{itemize}
\item{redefinition: a keyword is defined a second time for the child class only}
\item{polymorphism: an existing method can be redined with a slightly different signature}
\end{itemize}

From the perspective of the source code, a single keyword like \lstinline{owner} can refer to different storage slot depending on its context.
Only the bytecode makes a clear difference.

\subsection{Evasion Targets}

This technique is a refinment of the previous one: it will work on more targets.

\subsubsection{Users}

The source code is even closer to a legitimate contract: even with a large userbase, it is likely to fly under the radar.

\subsubsection{Reviewers}

The interpretation of the source code is subtle, and reviewing the bytecode is very time consuming.
So even people with the required skills may not have enough incentive to check the code in depth, outside of paid audits.

\subsection{Samples}

\subsubsection{Attribute Overwriting}

In section \emph{3.2.2}, the paper \href{\urlpaperartofthescam}{The Art of the scam} shows an example of inheritance overriding with \lstinline{KingOfTheHill} :

\begin{lstlisting}[language=Solidity]
contract KingOfTheHill is Ownable {
    address public owner; // different from the owner in Ownable

    function () public payable {
        if(msg.value > jackpot) owner = msg.sender; // local owner
        jackpot += msg.value;
    }
    function takeAll () public onlyOwner { // contract creator
        msg.sender.transfer(this.balance);
        jackpot = 0;
    }
}
\end{lstlisting}

In the modifier on \lstinline{takeAll}, the \lstinline{owner} points to the contract creator.
It is at storage slot 1, while the fallback function overwrites the storage slot 2.

In short, sending funds to this contract will never make you the actual owner.

\subsection{Detection \& Countermeasures}

\subsubsection{Source Code}

While subtle for the human reader, tools can easily scan the sources for duplicate definitions and polymorphism.

Since the whole point is to advertize for a functionality with the sources, they will be available.
However, the bytecode does not provide any information on this class of evasion.

This is where static analysis tools like \href{\urlcodeslither}{Slither} shine.
It has a \href{\urlcodeslithervariableshadowing}{specialized detector for keyword shadowing}.
