\section{Fake Standard Implementation} \label{sec:fake-implementation}

\subsection{Overview}

Such contracts borrow the function and class names from industry standards (OpenZeppelin, ERC, etc), but the code inside is actually different.

The malicious contracts generally pretend to be:

\begin{itemize}
\item{proxies: but the implementation is either not used or different from the ERC-1967 proxy}
\item{tokens: but the functions behave differently than ERC-20 / 721 / 1155}
\end{itemize}

Most often, the code will be very close to correct and vary only on key aspects.

\subsection{Evasion Targets}

\subsubsection{Users}

Few users actually check the code, so having a valid front is enough.

\subsubsection{Block explorers}

For now, block explorers have fixed models for proxies: they will show the address matching the ERC standards even if the contract actually uses another address.

\subsection{Samples}

\subsubsection{Fake EIP-1967 Proxy}

The \href{\urlstandardeipproxy}{standard EIP-1967} has pointers located in specific storage slots.
In particular, slot number \lstinline{0x360894a13ba1a3210667c828492db98dca3e2076cc3735a920a3ca505d382bbc} holds the address of the logic contract.

These pointers can be kept null or target a random contract, while the proxy actually uses another address.
A minimal example was given at \href{\urlvideomasqueradingcode}{DEFI summit 2023}:

\begin{lstlisting}[language=Solidity]
function _getImplementation() internal view returns (address) {
    return
        StorageSlot
            .getAddressSlot(bytes32(uint256(keccak256("eip1967.fake")) - 1)).
            .value;
}
\end{lstlisting}

Etherscan will show some irrelevant contract, giving the impression it is legit.

\subsubsection{Fake ERC20 Token}

Many phishing operations deploy fake tokens with the same symbol and name as the popular ones.

For example, \href{https://etherscan.io/address/0x5ed7ca349efc40550eecef4b288158fb2b9f12de}{this contract} is spoofing the USDC token.
It was used in \href{https://explorer.phalcon.xyz/tx/eth/0x7448178a8a03a0f1f298b697507f0e9172eacf1d32d422f48d0345c19c76eba3?line=33}{this phishing transaction}.

\subsection{Detection \& Countermeasures}

Several sources can be monitored, depending on the standard that is being spoofed:

\begin{itemize}
\item{Storage: comparing the target of \lstinline[language=Solidity]{delegateCall} to the address in the ERC storage slots}
\item{Events: changes to the address of the logic contract should come with an \lstinline[language=Solidity]{Upgraded} event}
\item{Bytecode: the implementation of known selectors can be checked against the standard's reference bytecode}
\end{itemize}
