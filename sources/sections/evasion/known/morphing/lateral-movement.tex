\section{Lateral Movement} \label{sec:lateral-movement}

\subsection{Overview}

After being detected, attackers can either improve their scheme... Or just rinse and repeat!
This is a very basic and widespread method.

More specifically, attackers can just:

\begin{itemize}
\item{create new EOA addresses}
\item{deploy several instances of their contracts}
\end{itemize}

\subsection{Evasion Targets}

\subsubsection{Block Explorers}

Many block explorers allow users to \href{https://etherscan.io/address/0x00000c07575bb4e64457687a0382b4d3ea470000}{tag addresses}, especially scams.

This is a manual process, so new addresses have to be discovered and tagged, even exact duplicates.

\subsubsection{User Tools}

This simple trick will get attackers past the blacklists of wallets and firewalls, for a time.

\subsection{Samples}

Fake tokens have been deployed in numerous phishing scams.
This particular USDT variant has \href{https://etherscan.io/find-similar-contracts?a=0xA15B3d31F1f5D544933C35eB00568Ead238B4f63&m=low&ps=25&mt=1}{412 siblings in ETH}.

\subsection{Detection \& Countermeasures}

\subsubsection{Bytecode}

Signatures of the attacking contract can be indexed in a database, so that when a new sample surfaces it will be instantly found.

\subsubsection{Graph Analysis}

The secondary addresses will most likely interact with their siblings / parent at some point.
In particular the collected funds may be redirected to a smaller set of addresses for cashout.

Graph analysis would propagate its suspicions from parent to child nodes.
