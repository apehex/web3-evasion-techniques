\section{Metamorphism} \label{sec:metamorphism}

\subsection{Overview}

So far, we've seen how to (re)deploy contracts at random addresses.
Given the size of the address space, there is close to no chance a contract is deployed in place of another one.
So contracts are usually thought as immutable.

However, the more recent opcode \lstinline{CREATE2} has a deterministic / known outcome.
It can be computed offline with:

\begin{lstlisting}[language=Solidity]
address(
    uint160(                        // downcast to match the address type.
        uint256(                    // truncate the upper digits.
            keccak256(              // compute the CREATE2 hash using 4 inputs.
                abi.encodePacked(   // pack all inputs to the hash together.
                  hex"ff",          // start with 0xff to distinguish from RLP.
                  sender,           // address of the contract calling CREATE2.
                  salt,             // some arbitrary salt.
                  hash              // the Keccak hash of the code to deploy.
                )
            )
        )
    )
);
\end{lstlisting}

This exerpt is adapted from \href{\urlcodemetamorphicapi}{0age's API}.
This tool allows to redeploy contracts at the same address in 3 transactions (A, B and C in the diagram):

\begin{tikzpicture}

\node (initcode) [io] {Init\\Code};
\node (implcode) [io, right=2cm of initcode] {Implementation\\Code};
\node (data) [io, right=2cm of implcode] {Data\\(Address)};

\node (factory) [block, below=2cm of implcode, inner sep=4mm] {Contract\\Factory};
\node (implementation) [block, below=6cm of initcode, inner sep=4mm] {Implementation\\Contract};
\node (mutant) [block, below=6cm of data, inner sep=4mm] {Metamorphic\\Contract};

\draw [arrow] ([xshift=0mm]initcode.270) -- +(0,-1cm) -- (factory.145);

\draw [arrow] ([xshift=0mm]implcode.270) -- ([xshift=0mm]factory.90);
\draw [arrow] ([xshift=0mm]factory.180)  -| node[label, near start, above] {B.2 create} (implementation.90);
\draw [arrow] ([xshift=-4mm]factory.270) |- node[label, near start, left] {B.3 store\\address} ([yshift=4mm]implementation.0);
\draw [arrow] ([xshift=0mm]factory.0)  -| node[label, near start, above] {B.4 create2} ([xshift=-4mm]mutant.90);
\draw [arrow] ([yshift=4mm]mutant.180) -| node[label, near end, right] {B.5 query\\address} ([xshift=4mm]factory.270);
\draw [arrow] ([yshift=-4mm]mutant.180) -- node[label, midway, below] {B.6 copy code} ([yshift=-4mm]implementation.0);

\draw [arrow] ([xshift=4mm]data.270) -- ([xshift=4mm]mutant.90);
\draw [arrow] ([xshift=0mm]mutant.270) -- node[label, midway, right] {C.2 selfdestruct} +(0,-2cm);

\node[label, anchor=east] at ([yshift=-12mm]initcode) {A.1 deploy};
\node[label, anchor=west] at ([yshift=-12mm]implcode) {B.1 send};
\node[label, anchor=west] at ([xshift=4mm, yshift=-12mm]data) {C.1 send};

\end{tikzpicture}


These steps have been outlined by Michael Blau in the article following the \href{\urlarticlemetamorphismtool}{publication of his detection tool}. 
This technique has many variants, in particular the init code in the API can be tweaked.

This init code is a small creation bytecode that is deployed using \lstinline{CREATE2}.
The same init code is used on every deployment of the metamorphic contract, which guarantees that it will be published at a fixed address.
The init code then copies the actual implementation code from another location giving birth to a new variant of the metamorphic contract.

\subsection{Evasion Targets}

\subsubsection{Users}

Few users are aware that contracts can change.
Checking a project once is already demanding, it is even less likely users will double check later.

\subsubsection{Reviewers}

Metamorphism allows to turn a totally legitimate contract into anything.
The first version may pass all the security checks, even though the presence of \lstinline{CREATE2} and \lstinline{SELFDESTRUCT} may raise some concern.

\subsection{Samples}

Metamorphism is commonly used by \href{\urltxmevbotmutation}{MEV bots}.

0age has also deployed several demonstration contracts, even \href{\urladdressmetamorphicfactory}{on the mainnet}.

\subsection{Detection \& Countermeasures}

Instead of characterizing a given address, it is more efficient to detect the mutation as it happens on each of the 3 key transactions depicted in the overview \ref{sec:metamorphism}.

\subsubsection{On Factory Deployment}

The process can be detected as soon as the \href{\urltxmetamorphismstepone}{factory deployment}.
The bytecode of the deployed contracts can be disassembled and scanned for signs of metamorphic code.

\begin{tikzpicture}

\node (traces) [io] {TX Traces};

\node (extract) [block, below=1cm of traces, inner sep=4mm] {Parse\\Creation Traces};
\node (data) [io, below=1cm of extract] {Addresses, Bytecodes};
\node (index) [block, right=4cm of data, inner sep=4mm] {Index};

\node (disassemble) [block, below=1cm of data, inner sep=4mm] {Disassemble};

\node (opcode) [decision, below=1cm of disassemble] {CREATE2?};
\node (init) [decision, below=1cm of opcode] {Init Code?};

\node (negative) [io, right=3cm of opcode] {0};
\node (unknown) [io, right=3cm of init] {0.5};
\node (positive) [io, below=2cm of init] {0.8};

\node (c1) [container, fit=(extract) (init), inner xsep=12mm, inner ysep=8mm, xshift=4mm, yshift=-2mm] {};
\node (t1) [label, above=2mm of c1, xshift=2cm] {Factory?};

\node (c2) [container, fit=(negative)] {};
\node (t2) [label, above=2mm of c2] {Negative};

\node (c3) [container, fit=(unknown)] {};
\node (t3) [label, above=2mm of c3] {Unkown};

\node (c4) [container, fit=(positive)] {};
\node (t4) [label, below=2mm of c4] {Positive};

\draw [arrow] (traces.south) -- (extract.north);
\draw [arrow] (extract.south) -- (data.north);
\draw [arrow] (data.south) -- (disassemble.north);
\draw [arrow] (data.east) -- (index.west);
\draw [arrow] (disassemble.south) -- (opcode.north);
\draw [arrow] (opcode.south) -- (init.north);
\draw [arrow] (opcode.east) -- node[label, near start, above] {No} (negative.west);
\draw [arrow] (init.south) -- node[label, near start, right] {Yes} (positive.north);
\draw [arrow] (init.east) -- node[label, near start, above] {No} (unknown.west);

\end{tikzpicture}


\subsubsection{On Mutant Metamorphing}

The mutation is the most telling transaction, as you can see \href{\urltxmetamorphismsteptwo}{in this example}.

The detection is very similar to the process described in the previous section.
Instead of looking for the indicators in the bytecode, they can directly be viewed in the execution traces.

At this stage the address of the morphing contract is known, a simple code-diff will give a definitive answer.

\begin{tikzpicture}

\node (traces) [io] {TX Traces};

\node (extract) [block, below=1cm of traces, inner sep=4mm] {Parse\\CREATE2 Traces};
\node (data) [io, below=1cm of extract] {Addresses, Inputs};
\node (index) [block, right=4cm of data, inner sep=4mm] {Index};

\node (codediff) [block, below=1cm of data, inner sep=4mm] {Code diff};
\node (diffinit) [decision, below=1cm of codediff] {Code $\neq$ Init?};

\node (scrape) [block, below=1cm of diffinit, inner sep=4mm] {Scrape History};
\node (changed) [decision, below=1cm of scrape] {Code Changed?};

\node (negative) [io, right=3cm of diffinit] {0};
\node (unknown) [io, right=2.8cm of changed] {0.5};
\node (positive) [io, below=2cm of changed] {1};

\node (c1) [container, fit=(extract) (changed), inner xsep=12mm, inner ysep=8mm, xshift=4mm, yshift=-2mm] {};
\node (t1) [label, above=2mm of c1, xshift=2cm] {Mutant?};

\node (c2) [container, fit=(negative)] {};
\node (t2) [label, above=2mm of c2] {Negative};

\node (c3) [container, fit=(unknown)] {};
\node (t3) [label, above=2mm of c3] {Unkown};

\node (c4) [container, fit=(positive) (positive)] {};
\node (t4) [label, below=2mm of c4] {Positive};

\draw [arrow] (traces.south) -- (extract.north);
\draw [arrow] (extract.south) -- (data.north);
\draw [arrow] (data.south) -- (codediff.north);
\draw [arrow] (data.east) -- (index.west);
\draw [arrow] (codediff.south) -- (diffinit.north);
\draw [arrow] (diffinit.south) -- (scrape.north);
\draw [arrow] (scrape.south) -- (changed.north);
\draw [arrow] (diffinit.east) -- node[label, near start, above] {No} (negative.west);
\draw [arrow] (changed.south) -- node[label, near start, right] {Yes} (positive.north);
\draw [arrow] (changed.east) -- node[label, near start, above] {No} (unknown.west);

\end{tikzpicture}


\subsubsection{On Mutant Suicide}

\href{\urltxmetamorphismstepthree}{Self-destructions} are fairly rare and can be the triggering factor of further analyses.
They can be directly detected using traces.
