\section{Anti-Reversing} \label{sec:anti-reversing}

\subsection{Overview}

Smart contracts hold and process key information they may want to keep private:

\begin{description}
\item[selectors]{as a whole, the interface reveals the parents of a given contract and gives an idea of its use}
\item[addresses]{addresses can designate the target of a hack, or recipients of a coin mixing operation for example}
\item[calldata]{it is a rich source of information that be used to understand the behavior of a contract}
\end{description}

Anti-reversing techniques aim at concealing or distorting these informations.
And they can go further by obfuscating the contract logic and generally interfering with RE tooling.

The application of such anti-reversing methods serves multiple purposes.
It can protect secret data, make it harder to find and / or exploit vulnerabilities, or hide the contract's intended functionality.

However, obfuscation is rarely used with good intentions: it is a red flag for security researchers.
Even though these methods hinder analysis, they may in turn motivate it!

\subsection{Evasion Targets}

\subsubsection{Tools}

Static analysis often employ common and imperfect methods that are both known and can be gamed.

\subsubsection{Reviewers}

Malicious agents can try can and delay / dissuade security researchers by obfuscating their smart contracts.

\subsubsection{MEV Researchers}

MEV researchers are very protective of their secret sauce: to prevent their competition from gaining insights, the bots are sometimes obfuscated.

\subsection{Samples}

\subsubsection{Scrambling Constants}

A basic technique to extract data from a contract is to disassemble it and look for \lstinline{PUSH}.
\lstinline{PUSH4} for selectors, \lstinline{PUSH20} for addresses, etc.

The MEV searcher DeGatchi mentions a few tricks \href{\urlarticleobfuscationinterface}{in his serie on obfuscation}.

\begin{itemize}
\item{several values can be packed into larger types, for example selectors into a \lstinline{bytes32}}
\item{data can be split in several parts and later reconstructed}
\item{real values can be computed from shifted / offset / arithmetically modified values, which are the only ones directly visible}
\end{itemize}

For example an address could be hidden in a bytes32:

\begin{lstlisting}
PUSH32 0x699694baab565c1557ace8cade5a7c71deadbeef00
PUSH1 0x08 
SHR
\end{lstlisting}

\subsubsection{Selector Obfuscation}

The naïve parsing of the contract of \href{\urladdressmevbot}{this MEV bot} will only yield \lstinline{6c0ffcda} and \lstinline{cabcfc90}.

But the selector \lstinline{0xf5c58cda} called to \href{\urltxmevhack}{front-run the hack on Wise Lending} does not appear in clear.
Actually it's totally ignored, the fallback is called and the value \lstinline{0xf5c58cda} is not used.

\subsubsection{Keeping Key Data Out Of The Bytecode}

The exploit contract of DappSocial didn't have the address of its target in the bytecode.

Even if someone identified the contract as malicious, it was very hard to guess the target in the 108s before the \href{\urltxdappsocialexploit}{attack transaction}.
