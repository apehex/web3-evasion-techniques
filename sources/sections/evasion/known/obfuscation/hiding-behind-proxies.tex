\section{Hiding Behind Proxies} \label{sec:hiding-behind-proxies}

\subsection{Overview}

Malicious contracts simply use the EIP-1967 \cite{eip-1967} specifications to split the code into proxy and logic contracts.

\subsection{Evasion Targets}

\begin{description}
\item[Etherscan]{the proxy contracts are often standard and will be validated by block explorers}
\item[users]{most users rely on block explorers to trust contracts}
\item[reviewers]{the source code for the logic contract may not be available: reversing and testing EVM bytecode is time consuming}
\end{description}

\subsection{Samples}

This \href{}{phishing contract} has its \href{}{proxy contract verified} by Etherscan.

While its logic contract is only available as \href{}{bytecode}.

\subsection{Detection \& Countermeasures}

Since it comes from Ethereum standards, this evasion is well-known and easy to detect.

However it is largely used by legitimate contracts, it is not conclusive by itself.

\begin{description}
\item[proxy patterns]{proxies can be identified from the bytecode, function selectors, storage slots of logic addresses, use \lstinline{delegateCall}, etc}
\item[block explorer]{the absence of verified sources is a stronger signal (to be balanced according to contract activity and age)}
\item[bytecode]{the bytecode of the logic contract can still be further analyzed}
\end{description}
