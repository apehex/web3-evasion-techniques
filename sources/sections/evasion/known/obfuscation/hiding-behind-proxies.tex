\section{Hiding Behind Proxies} \label{sec:hiding-behind-proxies}

\subsection{Overview}

Malicious contracts simply use proxy standards like the \href{\urlstandardeipproxy}{EIP-1967} specifications to split the code into proxy and logic contracts.

\subsection{Evasion Targets}

\subsubsection{Block Explorers}

Even when redirecting to malicious contracts, the proxy contracts themselves are often standard.
Block explorers will validate them and give a false sense of legitimacy.

\subsubsection{Users}

Most users rely on block explorers to trust contracts.

\subsubsection{Reviewers}

The source code for the logic contract will most likely not be available:
reversing and testing EVM bytecode is time consuming.

\subsection{Samples}

This \href{}{phishing contract} has its \href{}{proxy contract verified} by Etherscan.
While its logic contract is only available as \href{}{bytecode}.

\subsection{Detection \& Countermeasures}

Since it comes from Ethereum standards, this evasion is well-known and easy to detect.
However it is largely used by legitimate contracts, it is not conclusive by itself.

\begin{itemize}
\item{proxy patterns: proxies can be identified from the bytecode, function selectors, storage slots of logic addresses, use of \lstinline{delegateCall}, etc}
\item{block explorer: the absence of verified sources is a stronger signal (to be balanced according to contract activity and age)}
\item{bytecode: the bytecode of the logic contract should be further analyzed}
\end{itemize}
