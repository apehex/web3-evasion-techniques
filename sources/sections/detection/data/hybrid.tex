\section{Hybrid Analysis} \label{sec:data-hybrid}

Zooming out from the perspective of a single smart contract, the blockchain can be considered as a whole.
This is a mix of the static data across all addresses and the dynamic data generated across time and addresses.

Rather than going over all the data sources again, this section offers new angles from which they can be considered.

\subsection{Statistics}

The activity of a single address over time can be broken-down with statistics.

They can combine static and dynamic analyses by bringing out which functions / events are actually triggered and filtering out irrelevant code.

It will add to the previous analyses and weight all the smart contract actions with their frequency.
This temporal profile can be compared with other known contracts.

Independent transactions take perspective: are the interactions between addresses repeated?
Does the behavior of the contract change at any point?

\subsection{Graph Theory} \label{sec:hybrid-graph}

Graph theory will perform the same type of analysis than statistics while retaining more of the structure of the blockchain.

Indeed, the blockchain can be viewed as a graph with addresses as the nodes and transactions as the vertices.
The tricky part is to decide which specific metric will be used on nodes and vertices.

Even simple labeling schemes, like the transaction amount, will help to inspect the flow  of cash \& tokens.

Graph analysis can also be used to cluster the address space and show the similarities between contracts.

To fool these meta indicators, attackers may add legitimate use \& traffic to their contracts. 

\subsection{Symbolic Fuzzing}

Standard dynamic analysis will explore only a few execution paths during fuzzing.
Even the historical log of transactions will not show all the possible interactions with a contract.

The goal of symbolic analysis is to test all the execution branches and make the other detection techniques more exhaustive.

Symbolic testing has been adapted to Ethereum  \href{\urlcodehoneybadger}{HoneyBadger} leverage symbolic testing to explore all the execution paths.

This technique has known flaws: in particular, the number of conditional branches can be exponantially increased, leading to \href{\urlarticlepathexplosion}{path explosion}.

\subsection{Machine learning}

Machine learning can be used to achieve all of the above.

The ML models add a layer of abstraction that make the detection inherently more robust to small variations and improvements from the attackers.
They will also find new samples even when they were not exactly accounted for.

Tricky attackers may try and poison the models or flood the inputs with irrelevant data.
