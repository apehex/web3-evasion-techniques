\section{Static Analysis} \label{sec:data-static}

Outside of execution, the blockchain acts as a cold storage.
In this first context, the detection methods are called "static analysis".

\subsection{Creation Metadata}

The block and transaction objects hold a lot of data related to the infrastructure of the blockchain.
These informations, like \lstinline[language=Solidity]{block.difficulty} or \lstinline[language=Solidity]{block.gaslimit}, can be safely ignored when considering the smart contracts.

Other details like the contract's creator, the balance, the creation timestamp and associated Ether provide a context to the whole analysis.

\subsubsection{Contract's creator}

The values \lstinline[language=Solidity]{msg.sender} \& \lstinline[language=Solidity]{tx.origin} of the transaction that created the contract tell us who did it.
It is like having an IP: the addresses can be indexed to follow the activity of known attackers.

In turn, bad actors can simply use new "external owned accounts" (EOA) and redeploy / upgrade their contracts.

\subsubsection{Creation Cost}

The product of the gas price and gas used gives the cost of the smart contract deployment.
This gas consumption is related to the intensity of the processing involved.

EVM operation costs have a wide range and some key operations are especially pricy.
Contract activity can be differentiated based on this value: it is a built-in high level profiling tool.

\subsection{Compilation Metadata}

Similarly to traditional binaries, smart contracts are compiled into bytecode.
The settings used during the compilation are described in \href{\urldocssoliditymetadata}{a JSON file}.

The hash of these metadata may be appended to the bytecode:
it is actually an ID, which can be used to retrieve the metadata and possibly the sources from a IPFS.

In particular, the configuration of the optimizer is specified: the exact binary output of the compilation will vary according to these settings.
These informations can be used to adapt static analysis patterns to a specific target.

\subsection{Bytecode}

The main product of the compilation is the bytecode deployed on the mainnet.
It has several sections which can be parsed: OpenZeppelin wrote an in-depth article on the \href{\urldiagrambytecode}{structure of smart contract bytecode}.

In itself, providing only the bytecode (and not the sources) is already a layer of obfuscation.
But it is always available and has all the logic of the smart contract.

\subsubsection{Function Selectors}

Functions are not called by name, but by their selector.
And the selectors are hashes computed on the signature, like \lstinline[language=Solidity]{transfer(address,uint256)}:

\begin{lstlisting}[language=Python]
Web3.keccak(text='transfer(address,uint256)').hex().lower()[:10]
# '0xa9059cbb'
\end{lstlisting}

The list of selectors for all the function in the bytecode is \href{\urldiagrambytecode}{found in its hub}.

Keeping an \href{\urlwebindexselectors}{updated index of all known selectors} allows to go back from hash to signature.
It gives a lot of insight on the expected behavior of a contract.

On the other hand, nothing prevents malicious actors from \href{https://www.4byte.directory/signatures/?bytes4_signature=0xa9059cbb}{naming their functions as they please}.

\subsubsection{Function Bodies}

Of course, execution requires instructions: the function bodies implement the logic of the contract.

Just like binaries, they can be \href{\urlarticlereversingcontract}{reversed and analysed statically}.
This opens the way for pattern matching and manual reviews of the code.

However, these processes can be hindered with code stuffing and other techniques like packing (encryption, compression, etc).

\subsubsection{Constructor}

The smart contract constructor is not included in the bytecode deployed on the blockchain.
It is called once to initialize the contract state and generate the final code that will sit on the blockchain.

So it can be found in the data of the \href{https://etherscan.io/tx/0xd66169d4a5feaceaf777b9949ad0e9bc5621a438846a90087e50a5d7b9b0ad1e}{transaction that created the contract}.
Or in the source code, if provided (discussed below).

The constructor sets storage slots, which hold values that can totally change the behavior of the contract.
Admin privileges can act as a backdoor and enable rug pulls for example.

Attackers will try and sneak data into the contract's state.

\subsubsection{Opcode Sequence}

\href{\urlcodeevmdasm}{Bytecode can be interpreted as a language}, giving a level of abstraction to the analysis.
Indeed, different hex bytecodes can achieve the same result.
It is easier to get the high level logic from the sequences of opcodes than from raw and specific hex chunks.

But disassembling is not an exact science and it can be made even harder by classic techniques like \href{\urlpapereshield}{anti-patterns}.

\subsection{Source code}

First, source code is not always available: the blockchain itself doesn't hold it, it has to be supplied to third party services, e.g. block explorers.

With it, code review is humanly possible and reverse engineering becomes easier.
Sources help to understand new attacks, but are too time consuming to provide live intelligence.

Solidity can be misleading because of its ambiguities and \href{\urldocssoliditybugs}{bugs}.
Attackers will take advantage of the imprecision in the tools and the limited resources of human reviewers.
